\documentclass[10pt,a4paper]{article}
\usepackage[utf8]{inputenc}
\usepackage{graphicx}
\graphicspath{ {./images/} }

\usepackage{color}   
\usepackage{hyperref}
\usepackage{listings}
\usepackage[utf8]{inputenc}
\usepackage{amsmath}
\usepackage{amsfonts}
\usepackage{amsthm}
\usepackage{german}

\usepackage[onehalfspacing]{setspace}


\let\stdsection\section
\setlength{\parindent}{0in} 


% Useful packages
\usepackage{amsmath}
\usepackage{graphicx}
\usepackage[colorlinks=true, allcolors=blue]{hyperref}

\title{Neuronale Netze in der Bildverarbeitung Versuch 1}
\author{Julian Schmidt, Vincenz Forkel}

\begin{document}
\maketitle

\section{Einführung}
Ziel des ersten Versuchs des Praktikums Neuronale Netze in der Bildverarbeitung im Wintersemester 2021/22 ist es Grundlagen des Aufbaus und der Arbeit mit Neuronalen Netzen kennen zu lernen.\\
Konkret wird hier der MNIST-Datensatz verwendet, für den ein Neuronales Netz darauf trainiert werden soll Ziffern zwischen 0 und 9 zu klassifizieren.\\

Der MNIST (Modified National Institute of Standards and Technology database) ist eine öffentlich verfügbare Datenbank von handgeschriebenen Ziffern.\\
Wobei jede Ziffer als 28 × 28 Pixel großes Graustufen-Bild gespeichert ist.\\

Verwendet wird hierfür die Script Sprache Matlab
\newpage
\section{Netzwerk Architektur}
Um die Architektur des Netzwerkes zu erstellen wurde die Matlab-Toolbox für Neuronale Netze verwendet.

\lstinputlisting{layers.m}

Wie hier zu sehen ist besteht das Netz aus zwei 'fully connected layer', einem 'relu layer', einem 'softmax layer' und einem 'classification layer'.


\end{document}
